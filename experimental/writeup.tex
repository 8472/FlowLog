

\section{Surface Syntax}

A Flowlog program consists of a pair $(R, L)$ 

Together the \fl{incoming}, \fl{outgoing}, and \fl{remote}
declarations define an \emph{interface} for the module.

\subsection{Example}

\begin{flowlog}

  incoming stolen_laptop_report {mac} -> stolen_laptop_report(mac);
  incoming stolen_laptop_cancel {mac} -> stolen_laptop_cancel(mac);
  incoming timer_expired {id} -> timer_expired(id);
  incoming builtin packet -> EVpacket
  incoming builtin switchToPort -> EVswitchToPort

  outgoing builtin forward <- DOforward
  outgoing builtin emit <- DOemit
  
  outgoing stolen_laptop_seen {mac, swid, time} @ 127.0.0.1 5050 
    <- DOnotify_police(mac, swid, time);
  outgoing start_timer {seconds, id} @ 127.0.0.1 9091 
    <- DOnotify_start_timer(seconds, id);

  query time { X } @ 127.0.0.1 9091 <-> 
    GETtime(X); 
  query getroute {const dlDst, const swid, outpt} <-> 
    GETroute(dlDst, swid, outpt);

  query add {const X, const Y, Z} @ 127.0.0.1 1234 <-> 
    GETadd(X, Y, Z); 
  query sub {const X, const Y, Z} @ 127.0.0.1 1234 <-> 
    GETsubtr(X, Y, Z); 

  // Declare these to facilitate easier add/removal
  table stolen {mac};

\end{flowlog}

\question{HOW TO ALLOW HELPERS IF ALWAYS ``DO X'' IF...?}

\question{This example seems to be done in terms of tuple position,
  not field names. Why append (X,Y,Z)?}


\question{Get rid of Datalog syntax, inspired by SQL?}

\subsection{Desugaring Rules}




\section{Core Syntax}

A desugared Flowlog program is a pair $(D, L)$ in which $D$ is a set
of data-definitions and $L$ is a Datalog-with-negation program with
the following additional syntactic requirements:

\begin{enumerate}

\item{Every relation is of arity $\geq 1$. The first component of
  every relation is denoted its \emph{time argument}. }

\item{In every rule, every non-equality atomic formula in the body
  must use the distinguished variable $T$ for its time argument, and
  the rule's head must use the distinguished variable $S$ (short for
  ``successor'') for its time argument.}

\item{Every rule body must contain exactly one of the following atomic
  formulas: $(S = T)$ or $successor(T, S)$}

\end{enumerate}

\noindent These constraints borrow heavily from
\dedalus~\cite{}. Indeed, the language of $L$ is equivalent to
non-recursive \dedalus\ [[PROOF?]]. The de-sugaring process imposes
additional limitations.

\subsection{Example}

\begin{flowlog}
// State transitions are explicit. (This is boilerplate, can auto produce)
  stolen(T+1, mac) :- stolen(T, mac), not -stolen(T, mac);
  stolen(T+1, mac) :- +stolen(T, mac);

  +stolen(T, mac) :- stolen_laptop_report(T, rep), rep.mac = mac; // <-- collapse or expand types?
  -stolen(T, mac) :- stolen_laptop_cancel(T, rep), rep.mac = mac; 

//action BBpolice(pkt : packet, found: stolen_laptop_found) :- 

//  stolen(pkt.dlSrc),

//  found.mac = pkt.dlSrc,

//  found.swid = pkt.locSw,

//  BBTimer.time(found.time),

//  not ratelim(found.mac);



stolen_laptop_found(T, mac, swid, time) :-

  packetIn(T, pkt),

  stolen(T, pkt,dlSrc), not ratelim(T, mac),

  mac = pkt.dlSrc, swid = pkt.locSw, get_time(T, time);
\end{flowlog}


\subsection{Logical Semantics}

$\den{L}: DB \rightarrow DB$ by standard Datalog semantics. Since $L$
is a nonrecursive Datalog-with-negation program, it is stratified, and
thus the resulting $DB$ is uniquely determined. 

At first blush, there is no guarantee that the resulting $\den(L)(DB)$
is finite. Indeed, the inclusion of a $successor$ relation seems to
imply otherwise. 








\section{Interpreting Flowlog}

The Flowlog runtime responds to each incoming event $(etype, <v_1,
..., v_n>)$ as follows:

\begin{enumerate}
\item Increment event counter $T$;
\item Add the \edb\ fact \fl{successor(T, T+1)};
\item Add the event fact \fl{etype(T, v_1, ..., v_n)};
\item For each declared or built-in outgoing action $(A, <x_1, ...,
  x_m>, loc)$, issue the query \fl{A(x_1, ..., x_n)}, then pass each
  tuple $<c_1, ..., c_m>$ returned to $loc$. 
\end{enumerate}

\noindent Since every field must be positively constrained, query
response tuples contain only concrete constants. 

At each such step, the \edb\ is finite, since the $successor$ relation
contains no successor for time $T+1$.

The runtime handles the distinction between forwarding and emit, since
it knows the payload of the trigger packet.

\newcommand{\dedalus}{{\sc dedalus}$_0$}

We draw from both \dedalus\cite{} and the ``chain of fixpoints''
interpretation used by \ndlog\cite{}. We use several ideas from
\dedalus\ to give Flowlog's chain of evaluation a careful logical
semantics.



For blackbox relations---declared using the \kwremote\
keyword---our semantics assumes that the results of queries do
not change during a single evaluation step. This can be enforced
by adding transactional bookend functions to the blackbox API. 

Blackbox relations that are truly functional (i.e., do not change
at all) Flowlog supports the optional \kwpure\ keyword for
declarations.

To get \emph{multiple} results from a single blackbox in a single
evaluation step, one makes multiple declarations. For instance:

\begin{flowlog}
  remote nonce1 {X} @ localhost 9090;
  remote nonce2 {X} @ localhost 9090;
\end{flowlog}

\noindent two distinct invocations available in the \fl{nonce1}
and \fl{nonce2} relations.

To obtain a finer grain (for instance, a nonce for each outgoing
packet), add to the arity for salts:

\begin{flowlog}
action forward(pkt : packet, newpkt : packet) :-
  ...
  not pkt.locPt = newpkt.locPt,
  nonce(newpkt.locPt, X);
\end{flowlog}

Q: better to make the wrappers support this? Or some syntax to force
re-query for every <salt>?



---------------------------

LINQ Updates:

Also field-name based, not position based:

// select to prepare for update
Category category = 
  bookCatalog.Categories.Single( c => c.Name == "Programming Practices" );

// update
category.Name = "Technical Practices";

// commit
bookCatalog.SubmitChanges( );


LINQ INSERTIONS:

// create new row
Category category = new Category( ) { Name = "Java" };
// insert the row
bookCatalog.Categories.InsertOnSubmit( category );
// and commit
bookCatalog.SubmitChanges( );

LINQ deletions

Category category = bookCatalog.Categories.Single( c => c.Name == "Java" );
bookCatalog.Categories.DeleteOnSubmit( category );
bookCatalog.SubmitChanges( );


``create/select'' phase (1)

``x''

``commit'' phase / insertion (3) is unnecessary in Flowlog. Any transactions should be implicit.


can be done w/o transactional too

context.CustomerOrders(order => order.CustomerId == 255).DeleteAll();

var query = context.CustomerOrders.Where(order => order.CustomerId == 255);
query.UpdateAll(ord => ord.Set(o => o.IsDeleted, o => true));

The cool part here is the => --- the lambda operator.
passing a custom predicate. but for us, the predicates dont need lambdas. and insertions don't use. 







(1) I looked at 2 separate guides to LINQ updates. They look like this
(where => is a lambda expression):

var query = context.CustomerOrders.Where(order => order.CustomerId == 255);
query.UpdateAll(ord => ord.Set(o => o.IsDeleted, o => true));

So the conditions are flexible via => and the changes can be made at
any point in a program. But for our purposes---a rule-based language
with boolean conditions---I think that LINQ updates are quite similar
to ordinary SQL.

(2) I tried field-name based access, but it turned into a mess as soon
as I looked at joins. For instance, if I want to write the rule:

plus foo(X) :- R(X, Y), R(Y, Z). // self-join

we'd need something like:

R: rf1, rf2;  // declare fieldnames (not too bad)

plus foo(X) :- 
  [USE R AS R1, R AS R2] // 2 identifiers for self-join, same as in SQL
  R1.rf1 = X and
  R1.rf2 = R2.pf1
  
Ugh! So the example below uses position-based fields.

--------------------------------------------------------------------------------

A Flowlog program is a pair (D, R) where D is a set of data
declarations and R is a set of rules. Together, the data declarations
define the program's *interface*, and the rules define its *behavior*.

------------------------------------------------------------------
EXAMPLE DATA DEFINITIONS (Naive Stolen Laptop)
------------------------------------------------------------------

  // Incoming notifications: notification type name, field names.
  // Relation name for use in rules will be created automatically in desugaring.
  incoming stolen-laptop-report {mac};
  incoming stolen-laptop-cancel {mac};
  incoming timer-expired {id};  

  // Built-in notifications don't need to be declared. Packets, etc. are defined automatically.
  
  // When a tuple goes into DOnotify-police, send as a stolen-laptop-seen to localhost:5050
  outgoing stolen-laptop-seen {mac, swid, time} @ 127.0.0.1 5050 
    <- DOnotify-police;
  outgoing start-timer {seconds, id} @ 127.0.0.1 9091 
    <- DOnotify-start-timer;

  // Don't conflate the relation into which tuples go (e.g. DOnotify-police) with
  // notification type---we may want to be able to send the same kind of event
  // to multiple places! ("Campus police says to send all laptop MACs to port 5050,
  // but smartphone MACs should go to port 5051.")

  // GETtime (unary) is a relation provided via a black-box query
  // to localhost:9091 with id "time".

  remote time {t} @ 127.0.0.1 9091 -> 
    GETtime; 

------------------------------------------------------------------
EXAMPLE RULES (Naive Stolen Laptop)
POSITION-BASED FIELDS
------------------------------------------------------------------

// "ON" blocks for trigger events.
// For brevity, can have multiple rules 
// (DO, INSERT, DELETE, UPDATE, ...) per trigger block:
ON packet pkt:

  DO forward(new) WHERE
    Mac-learning.forward(new);

  DO stolen-laptop-found(mac, swid, time) WHERE  
    stolen(pkt.dlSrc) AND
    not ratelim(mac) AND    
    mac = pkt.dlSrc AND 
    swid = pkt.locSw AND
    GETtime(time);

ON stolen-laptop-cancel can:
  DELETE (mac) FROM stolen WHERE
    can.mac = mac;

ON stolen-laptop-report rep:
  INSERT (mac) INTO stolen WHERE
    rep.mac = mac;

------------------------------------------------------------------
Syntactic issues beyond this example:
------------------------------------------------------------------

// (A)
// Syntactic sugar for UPDATE. E.g. here is the mac-learning 
// state change in a single rule, rather than two:

ON packet pkt:
  UPDATE (sw,mac,pt) IN learned WHERE
    pkt.dlSrc = mac,
    pkt.locSw = sw,
    pkt.locPt = pt;

// (B)
// Syntactic sugar for multiple actions per rule. For instance, 
// tell the police *and* make a record if we see a stolen MAC.
// (Note support for shared variables across the actions.)

ON packet pkt:
  DO stolen-laptop-found(mac, swid, time),
  INSERT (mac, swid) INTO seen WHERE  
    stolen(pkt.dlSrc) AND
    not ratelim(mac) AND    
    mac = pkt.dlSrc AND 
    swid = pkt.locSw AND
    GETtime(time);

// (C)
// Helper relations, independent of triggers:

HELPER dangerous-mac(mac) IF
  learned(sw, mac, pt) AND
  learned(sw, mac, pt2) AND
  pt != pt2;

// Some applications of helpers can be sugared away, but it seems a
// good idea to allow them for code clarity reasons. They do smell
// different from the other kinds of rule, though: they aren't taking
// an *action* 

// (D)
// We may want to make people explicitly write "EXISTS X" in rules.
// It is very easy to use the wrong variable name and end up using
// a fresh existential by mistake. Consider, in the above:

HELPER dangerous-mac(mac) IF
  learned(sw, mac, pt) AND
  learned(sw, mac, pt2) AND
  pt1 != pt2;  // <--- typo! s/pt/pt1

// This would be caught if we required :

HELPER dangerous-mac(mac) IF
  EXISTS sw, pt, pt2 |
    learned(sw, mac, pt) AND
    learned(sw, mac, pt2) AND
    pt1 != pt2;  // 

ERROR >> [pt1] not in scope. (Better error message here.)

// But such a requirement could be cumbersome, and bloats the rules.

// (E)

// If we enforce EXISTS, we should have sugar for FORALL that
// automatically creates helpers.
















------------------------------------------------------------------
LEXICAL ALTERNATIVES
------------------------------------------------------------------
Possible synonyms for incoming: "trigger", "input", ...
Possible synonyms for outgoing: "event", "output", ...
Possible synonyms for remote: "external", "blackbox", "query" ...

"WHERE" could become "IF" or "WHEN".
"ON" could become "WHEN" 
etc.



////////
// trimmed


  // Declarations can make restrictions on how a BB query is used. 
  // For instance, we could say that add must never be invoked in a way
  // that will yield an infinite reply, and say it's pure functional, for 
  // optimization:

  pure remote add {const X, const Y, Z} @ 127.0.0.1 1234 -> 
    GETadd;

  // Multiple relations can be tied to a single black-box query, resulting in
  // multiple invocations. This lets developers easily obtain (e.g.) multiple
  // nonces. This declaration makes 2 different nonce values available via
  // different relations:

  remote nonce {const X, const Y, Z} @ 127.0.0.1 1234 -> 
    GETnonce1, GETnonce2;

  // (add and nonce aren't used in this program, but are here for illustration/discussion.)
