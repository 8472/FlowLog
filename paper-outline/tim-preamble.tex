\usepackage{xspace}

%%%%%%%%%%%%%%%%%%%%%%%%%%%%%%%%%%%%%%%%%%%
% Abbrieviations, small-caps acronyms, etc.
%%%%%%%%%%%%%%%%%%%%%%%%%%%%%%%%%%%%%%%%%%%

\newcommand{\fancyg}{\ensuremath{\mathcal{G}}\xspace}
\newcommand{\fancyu}{\ensuremath{\mathcal{U}}\xspace}
\newcommand{\fancyA}{\ensuremath{\mathcal{A}}\xspace}

\newcommand{\edb}{\textsc{edb}\xspace}
\newcommand{\idb}{\textsc{idb}\xspace}
\newcommand{\edbs}{\textsc{edb}s\xspace}
\newcommand{\idbs}{\textsc{idb}s\xspace}
\newcommand{\nat}{\textsc{nat}\xspace}
\newcommand{\ios}{\textsc{ios}\xspace}
\newcommand{\dmz}{\mbox{\sc{dmz}}}
\newcommand{\junos}{Jun\textsc{os}\xspace}
\newcommand{\bdd}{\textsc{bdd}\xspace}
\newcommand{\acl}{\textsc{acl}\xspace}
\newcommand{\ip}{\textsc{ip}\xspace}
\newcommand{\tcp}{\textsc{tcp}\xspace}
\newcommand{\ack}{\textsc{ack}\xspace}
\newcommand{\ftp}{\textsc{ftp}\xspace}
\newcommand{\repl}{\textsc{repl}\xspace}
\newcommand{\xacml}{\textsc{xacml}\xspace}
\newcommand{\sqs}{\textsc{sqs}\xspace}
\newcommand{\json}{\textsc{json}\xspace}
\newcommand{\iptables}{\textsc{ip}tables\xspace}
\newcommand{\xml}{\textsc{xml}\xspace}
\newcommand{\cnf}{\textsc{cnf}\xspace}
\newcommand{\dnf}{\textsc{dnf}\xspace}
\newcommand{\smt}{\textsc{smt}\xspace}
\newcommand{\sat}{\textsc{sat}\xspace}
\newcommand{\jvm}{\textsc{jvm}\xspace}

\newcommand{\etal}{\textit{et al.}\xspace}
\newcommand{\cf}{\textit{c.f.}\xspace}
\newcommand{\eg}{\textit{e.g.}\xspace}
\newcommand{\ie}{\textit{i.e.}\xspace}
\newcommand{\etc}{\textit{etc.}\xspace}

\newcommand{\dontcare}{don't-care\xspace}
\newcommand{\ju}{jointly universal\xspace}

%%%%%%%%%%%%%%%%%%%%%%%%%%%%%%%%%%%%%%%%%%%
% Math
%%%%%%%%%%%%%%%%%%%%%%%%%%%%%%%%%%%%%%%%%%%

% Notes-to-self while document is in progress
% Aside used to be something different
\newcommand{\mynote}[1]{\textbf{[TN: #1]}}
%\newcommand{\todo}[1]{\textbf{[!!! TODO: #1]}}
\newcommand{\aside}[1]{\textbf{[TN: #1]}}

% nitpicks are imported from other docs, etc. 
% Unnecessary detail, but left in source in case it's needed later.
\newcommand{\nitpick}[1]{}

\newcommand{\fancyP}{\mbox{$\mathcal{P}$}\xspace}
\newcommand{\fancyF}{\mbox{$\mathcal{F}$}\xspace}
\newcommand{\fancyR}{\mbox{$\mathcal{R}$}\xspace}

\newcommand{\mdl}[1]{\ensuremath{\mathbb{{#1}}}\xspace}
\newcommand{\mdlM}{\mdl{M}}
\newcommand{\mdlP}{\mdl{P}}
\newcommand{\mdlN}{\mdl{N}}
\newcommand{\mdlK}{\mdl{K}}
\newcommand{\mdlH}{\mdl{H}}
\newcommand{\mdlMchase}{\ensuremath{\mdlM_{chase}}\xspace}
\newcommand{\mdlMcore}{\ensuremath{\mdlM_{core}}\xspace}
\newcommand{\srt}{\ensuremath{A}}
\newcommand{\srts}{\ensuremath{\langle A_1 \dots A_n \rangle}}
\newcommand{\tuple}[1]{\mbox{$\langle #1\rangle$}}
\newcommand{\env}{\ensuremath{\eta}\xspace}
\newcommand{\envext}{\ensuremath{\hat{\eta}}\xspace}
\newcommand{\options}{\ensuremath{O}\xspace}
\newcommand{\upperbound}{\ensuremath{B^\uparrow}}
\newcommand{\lowerbound}{\ensuremath{B_\downarrow}}
\newcommand{\ppbe}{propositionalization by element\xspace}
\newcommand{\ppbt}{propositionalization by term\xspace}
\newcommand{\semfact}{semantic fact\xspace}
\newcommand{\semfacts}{semantic facts\xspace}
\newcommand{\transfunc}{\ensuremath{\mathbb{T}}}
\newcommand{\transfuncinv}{\ensuremath{\mathbb{T}^{-1}}}

\newcommand{\sentence}{\ensuremath{\sigma}}

\newcommand{\thy}{\ensuremath{\tau}}
\newcommand{\consist}{\ensuremath{\mathit{consist}}}

\newcommand{\store}{\ensuremath{\sigma}\xspace}
\newcommand{\formenv}{\ensuremath{\Gamma}\xspace}
\newcommand{\sorts}{\ensuremath{\mathcal{S}}\xspace}
\newcommand{\res}[1]{\textbf{#1}}
\newcommand{\iter}[1]{\mathcal{I}_{{#1}}}



% Vocabulary
%\newcommand{\vocab}{\ensuremath{\mathcal{V}}\xspace}  % already defined
\newcommand{\query}{\ensuremath{\alpha}\xspace}
\newcommand{\queryc}{\ensuremath{\sigma_\alpha}\xspace}
%\newcommand{\sig}{\ensuremath{\Sigma}\xspace}
\newcommand{\sigq}{\ensuremath{\Sigma_{\alpha}}\xspace}
\newcommand{\sigm}{\ensuremath{\Sigma_{\mdlM}}\xspace}
\newcommand{\sigqm}{\ensuremath{\Sigma_{(\alpha, \mdlM)}}\xspace}
\newcommand{\sigorig}{\ensuremath{\Sigma_{orig}}\xspace}

\newcommand{\modelsof}{\ensuremath{\mathit{mod}}}
\newcommand{\answers}{\ensuremath{\mathit{answers}}}
\newcommand{\support}{\ensuremath{\mathit{support}}}
\newcommand{\covers}{covers\xspace}
\newcommand{\osepl}{order-sorted effectively-propositional logic}

%%%%%%%%%%%%%%%%%%%%%%%%%%%%%%%%%%%%%%%%%%%
% Environments
%%%%%%%%%%%%%%%%%%%%%%%%%%%%%%%%%%%%%%%%%%%

\newtheorem{defn}{Definition}[section]
\newtheorem{thm}{Theorem}[section]
\newtheorem{lemma}{Lemma}[section]
%\newtheorem{proposition}{Proposition}[section]
\newtheorem{corollary}{Corollary}[section]
%\newtheorem{algorithm}{Algorithm}[section]
\newtheorem{remark}{Remark}[section]


% Numbered by goal, not by section
\newtheorem{goal}{Research Goal}
\newtheorem{example}{Example}

\newcommand{\dhat}{\ensuremath{\hat{d}}\xspace}
\newcommand{\xvec}{\ensuremath{\overline{x}}\xspace}
\newcommand{\uvec}{\ensuremath{\overline{u}}\xspace}
\newcommand{\vvec}{\ensuremath{\overline{v}}\xspace}
\newcommand{\avec}{\ensuremath{\overline{a}}\xspace}
\newcommand{\bvec}{\ensuremath{\overline{b}}\xspace}
\newcommand{\cvec}{\ensuremath{\overline{c}}\xspace}
\newcommand{\dvec}{\ensuremath{\overline{d}}\xspace}
\newcommand{\tvec}{\ensuremath{\overline{t}}\xspace}

\newcommand{\yieldpt}{\ensuremath{\mathit{yield}_+}\xspace}
\newcommand{\yieldnt}{\ensuremath{\mathit{yield}_-}\xspace}
\newcommand{\potentpt}{\ensuremath{\mathit{potent}^T_+}\xspace}
\newcommand{\potentnt}{\ensuremath{\mathit{potent}^T_-}\xspace}
\newcommand{\refinements}{\ensuremath{\leq}\xspace}
\newcommand{\treach}{\ensuremath{\mathit{reach}_T}\xspace}

\newcommand{\posl}{\ensuremath{\mathit{pos}}\xspace}
\newcommand{\negl}{\ensuremath{\mathit{neg}}\xspace}
\newcommand{\lits}{\ensuremath{\mathit{lits}}\xspace}
\newcommand{\andlits}{\ensuremath{\bigwedge \lits}\xspace}
\newcommand{\andpos}{\ensuremath{\bigwedge \posl}\xspace}
\newcommand{\ets}{\ensuremath{\mathit{ets}}\xspace}
\newcommand{\reacht}{\ensuremath{\mathit{reach}_T}\xspace}

\newcommand{\tconstraints}{\ensuremath{\Sigma_t}\xspace}
\newcommand{\constraints}{\ensuremath{\Sigma}\xspace}
\newcommand{\stconstraints}{\ensuremath{\Sigma_{s,t}}\xspace}
\newcommand{\simconstraints}{\ensuremath{\overline{\Sigma}}\xspace}
\newcommand{\tvocab}{\ensuremath{\tau}\xspace} 
\newcommand{\tgd}{{\sc tgd}\xspace}
\newcommand{\egd}{{\sc egd}\xspace} 
\newcommand{\tgds}{{\sc tgd}s\xspace}
\newcommand{\egds}{{\sc egd}s\xspace}
\newcommand{\retract}{\hookrightarrow}
\newcommand{\quest}[1]{{\color{red}\textbf{#1}}}

\newcommand{\real}{\ensuremath{real}\xspace}
\newcommand{\action}{\ensuremath{\beta}\xspace}
\newcommand{\forces}{\ensuremath{\Vdash}\xspace}
\newcommand{\pimplies}{\ensuremath{\rightharpoonup}\xspace}
\newcommand{\emptyscenario}{\ensuremath{\Phi}\xspace}
\newcommand{\weakcf}{\ensuremath{\mathcal{F}}}
\newcommand{\pcf}{\ensuremath{\mathcal{F}_\leq}\xspace}
\newcommand{\ncf}{\ensuremath{\mathcal{F}_\geq}\xspace}

%%%%%%%%%%%%%%%%%%%%%%%%%%%%%%%%%%%%%%%%%%%
% Special formatting
%%%%%%%%%%%%%%%%%%%%%%%%%%%%%%%%%%%%%%%%%%%

% Allows special formatting for the different identifier types
\newcommand{\qterm}[1]{{\small{\texttt{#1}}}}
\newcommand{\varname}[1]{\emph{#1}}
\newcommand{\vocabname}[1]{\emph{#1}}
\newcommand{\polname}[1]{\texttt{#1}}
\newcommand{\decname}[1]{\emph{#1}}
\newcommand{\axiomname}[1]{\emph{#1}}
\newcommand{\sortname}[1]{\emph{#1}}
\newcommand{\relname}[1]{\qterm{#1}}

\def\refsec#1{Section~\ref{#1}}
\def\refdef#1{Definition~\ref{#1}}
\def\reflem#1{Lemma~\ref{#1}}
\def\refrem#1{Remark~\ref{#1}}
\def\refprop#1{Proposition~\ref{#1}}
\def\refthm#1{Theorem~\ref{#1}}
\def\refcor#1{Corollary~\ref{#1}}
\def\reffig#1{Figure~\ref{#1}}
\def\refex#1{Exercise~\ref{#1}}
\def\refeg#1{Example~\ref{#1}}
\def\refalg#1{Algorithm~\ref{#1}}

%% ==============
%% Time-stamping
%% ==============

% From Dan

%% compute the time in hours and minutes;
%% usage: \timestamp
\newcount\timehh\newcount\timemm \timehh=\time
\divide\timehh by 60 \timemm=\time
\count255=\timehh\multiply\count255 by -60 \advance\timemm by \count255
\newcommand{\timestamp}{
  {\protect\small\sl\today\ --
    \ifnum\timehh<10 0\fi\number\timehh\,:\,
    \ifnum\timemm<10 0\fi\number\timemm}}


\newcommand{\st}{\ensuremath{\; . \;}\xspace}
\newcommand{\squish}{\vspace*{-10pt}}
\newcommand{\pinch}{\vspace*{-5pt}}

%% Listed sets
\newcommand{\set}[1]{\ensuremath{ \{ #1 \} }} 
%% Set description
\newcommand{\dset}[2]{ \{ {#1} \mid {#2} \} } 
%% Set description with text as the description-part
\newcommand{\txtdset}[2]{ \{ {#1} \mid {\mbox{#2}} \} } 
%% Sequence description
\newcommand{\seq}[1]{ \langle {#1}  \rangle } 
\newcommand{\dseq}[2]{ \langle {#1} \mid {#2} \rangle } 
\newcommand{\empseq}{\ensuremath{\langle \rangle}}

